% !TeX root = tcolorbox.tex
% include file of tcolorbox.tex (manual of the LaTeX package tcolorbox)
\clearpage
\section{Library \mylib{breakable}}\label{sec:breakable}%
\tcbset{external/prefix=external/breakable_}%
The library is loaded by a package option or inside the preamble by:
\begin{dispListing}
\tcbuselibrary{breakable}
\end{dispListing}
This also loads the package |pdfcol|.

\subsection{Technical Overview}
The library \mylib{breakable} supports the automatic breaking of a |tcolorbox|.
This feature is enabled by \refKey{/tcb/breakable}
and disabled by \refKey{/tcb/unbreakable}.

{
\tcbset{colframe=Navy,colback=AliceBlue,fonttitle=\bfseries,
  watermark color=AliceBlue!85!Navy,enhanced}
If a |tcolorbox| is set to be \refKey{/tcb/breakable}, then the following
algorithm is executed:
\begin{enumerate}
\item The box content is read to a box register similar but not identical
  to the unbreakable case.
\item If the total box fits into the current page, it is shipped out visibly
  unbroken and the algorithm stops.
  \begin{tcolorbox}[title=Unbroken Box,watermark text=unbroken]
  The box.
  \end{tcolorbox}
\item Otherwise, it is checked if at least \refKey{/tcb/lines before break}
  of the upper box can be placed on the current page.
  If not, a page break is inserted and the algorithm goes back to Step 2.
\item Now, the \emph{break sequence} starts.
  The upper box part or the lower box part is split such that it fits
  into the current page. The fitting part is named \emph{first part} of
  the \emph{break sequence} and shipped out.
  \begin{tcolorbox}[title=Broken Box,watermark text=first,skin=enhancedfirst]
  The box.
  \end{tcolorbox}
\item
  If the remaining content of the total box
  fits into the current page, the algorithm continues with Step 7, else
  with Step 6.
\item
  The upper box part or the lower box part is split such that it fits
  into the current page. The fitting part is named \emph{middle part} of
  the \emph{break sequence} and shipped out.
  Then, the algorithm goes back to Step 5.
  \begin{tcolorbox}[watermark text=middle,skin=enhancedmiddle]
  The box.
  \end{tcolorbox}
\item
  The remaining part is named \emph{last part} of
  the \emph{break sequence} and shipped out. The algorithm stops.
  \begin{tcolorbox}[watermark text=last,skin=enhancedlast]
  The box.
  \end{tcolorbox}
\end{enumerate}
}

The algorithm takes care that the optional segmentation line never appears at
the end of a box. The optional lower box part is also checked to
have at least \refKey{/tcb/lines before break}.

\clearpage
In principle, all boxes of the \emph{break sequence} share the same geometric
parameters. The differences are:
\begin{itemize}
\item The given \refKey{/tcb/before} and \refKey{/tcb/after} values are
  used only before the \emph{first} and after the \emph{last} part
  of the \emph{break sequence}.
\item A special behavior between the parts of the \emph{break sequence} can
  be given by \refKey{/tcb/toprule at break},
  \refKey{/tcb/bottomrule at break},
  \refKey{/tcb/enlarge top at break by}, and
  \refKey{/tcb/enlarge bottom at break by}.
\item The \refKey{/tcb/skin} decides \emph{how} the \emph{first}, \emph{middle},
  and \emph{last} part look like. Actually, every part type has its own
  skin given by the options  \refKey{/tcb/skin first}, \refKey{/tcb/skin middle}, and
  \refKey{/tcb/skin last}. Typically, these options are set automatically by
  the main skin, see Subsection \ref{subsec:breaksequence} from
  page \pageref{subsec:breaksequence}.
\end{itemize}


\subsection{Limitations and Known Bugs}
\begin{itemize}
\item  The maximal total height of the upper and of the lower part
  of normal breakable |tcolorbox|es is about 65536pt (ca.\,2300cm)
  apiece. If such a part gets longer, the output will get buggy
  without warning.
  For very oversized boxes which are longer than 65536pt, use
  the \docValue{unlimited} value for  \refKey{/tcb/breakable}.
  With the \docValue{unlimited} setting,
  the applied algorithm has (virtually) no height limit for boxes, but
  very likely the compiler memory will have to be increased for boxes longer
  than 300 pages (depending on compiler settings and box content).
  But it is recommended to use \docValue{unlimited} for critical large boxes only.
  %,since there \emph{may} be a single interline space deviation (lost glue) around
  %every 2300cm, e.\,g.\ a \refCom{tcbline*} \emph{may} get lost.
\item You can nest an unbreakable |tcolorbox| inside another |tcolorbox|,
  even inside a breakable one.
  But you cannot not nest a breakable box inside a breakable box.
  The \refKey{/tcb/breakable} key for a nested box is ignored
  automatically\footnote{Until |tcolorbox| 3.04, the \refKey{/tcb/breakable} key
  was not ignored for nested boxes.}, i.\,e.\ inner
  boxes are always unbreakable.\par
  After all, in the unlikely case you really want to have the nested box to be breakable,
  use \refKey{/tcb/enforce breakable} for the nested
  box\footnote{\refKey{/tcb/enforce breakable} acts like \refKey{/tcb/breakable} until |tcolorbox| 3.04.}.
  \textbf{But, a breakable box inside a breakable box will usually give a mess.}
\item\tcbdocmarginnote{\tcbdocnew{2020-09-17}}
  Depending on the \LaTeX\ engine, if your text content contains some text
  color changing commands, your color may not survive the break to the next box.
  See the documentation for \refKey{/tcb/use color stack} for more information.
\item\tcbdocmarginnote{\tcbdocnew{2014-10-30}}
  The |perpage| option of the |footmisc| package is deliberately deactivated
  inside a breakable box since all footnotes are placed at the end
  of the box (possibly far away from the reference point).
\item\tcbdocmarginnote{\tcbdocnew{2016-02-15}}
  Making a box \refKey{/tcb/breakable} which actually is not broken creates
  a box which acts \emph{almost} like an unbreakable box. Visual differences
  are kept as indiscernible as possible, but can appear with certain
  \refKey{/tcb/before} and \refKey{/tcb/after} settings, especially, if there
  is an automatic page break before the box.
\item\tcbdocmarginnote{\tcbdocnew{2016-05-25}}
  Lua\TeX\ version 0.95 changes the behavior of the basic |\vsplit| (a bug?!)
  resulting in badly broken boxes. Thanks to Jeremy Engel,
  the \mylib{breakable} library contains a patch for this which
  also loads the the |ifluatex| package.
\end{itemize}


\clearpage
\subsection{Main Option Keys}
\begin{docTcbKey}[][doc updated=2017-02-01]{breakable}{\colOpt{=true\textbar false\textbar unlimited}}{default |true|, initially |false|}
  Allows the |tcolorbox| to be breakable. If the box is larger than the
  available space at the current page, the box is automatically broken
  and continued to the next page. All sorts of |tcolorbox| can be made
  breakable. It depends on the skin how the breaking looks like.
  If you do not know better, use \refKey{/tcb/enhanced} for breaking a box.
  The parts of the \emph{break sequence} are numbered
  by the counter |tcbbreakpart|.
  \begin{itemize}
  \item\docValue{false}: Sets the |tcolorbox| to be unbreakable.
  \item\docValue{true}: Breaks the |tcolorbox| from one page to another.
    The maximal total height of the upper and of the lower part is
    about 65536pt (ca.\,2300cm or ca.\,90 pages) apiece.
  \item\docValue{unlimited}: Experimental code for unlimited total height of
    breakable boxes.
    %There \emph{may} be a single interline space deviation (lost glue) around every 2300cm.
    For boxes longer than 300 pages (or even shorter ones) the
    compiler memory will have to be increased.
  \end{itemize}

\begin{dispListing}
% \usepackage{lipsum}  % preamble
\tcbset{enhanced jigsaw,colback=red!5!white,colframe=red!75!black,
  watermark color=yellow!25!white,watermark text=\arabic{tcbbreakpart},
  fonttitle=\bfseries}

\begin{tcolorbox}[breakable,title=My breakable box]
\lipsum[1-6]
\end{tcolorbox}
\end{dispListing}
\end{docTcbKey}
{\tcbusetemp}


\begin{docTcbKey}{unbreakable}{}{no value, initially set}
  Sets the |tcolorbox| to be unbreakable.
\end{docTcbKey}


\begin{docTcbKey}{enforce breakable}{}{no value}
  A |tcolorbox| inside a |tcolorbox| is automatically set to be unbreakable.
  Using \refKey{/tcb/breakable} on such an inner box has no effect.
  If one \emph{really} wants the inner box to be breakable, use \refKey{/tcb/enforce breakable}.
  \textbf{This will usually give a mess of shattered boxes. You are advised to not use this option.}\\
  Note that \refKey{/tcb/enforce breakable} has the functionality
  that \refKey{/tcb/breakable} had until package version 3.04
  and exists for backward compatibility.
\end{docTcbKey}


\begin{docTcbKey}[][doc updated=2018-07-26]{title after break}{=\meta{text}}{no default, initially empty}
  The \refKey{/tcb/title} is used only for the \emph{first} part of a
  \emph{break sequence}. Use |title after break| to create a heading line
  with \meta{text} as content for all following parts.
  Also see \refKey{/tcb/extras title after break} for formatting the title text.
\end{docTcbKey}


\begin{docTcbKey}{notitle after break}{}{no value, initially set}
  Removes the title line or following parts in a  \emph{break sequence} if set before.
\end{docTcbKey}


\begin{docTcbKey}{adjusted title after break}{=\meta{text}}{style, no default, initially unset}
  Works like \refKey{/tcb/adjusted title} but applied to \refKey{/tcb/title after break}.
\end{docTcbKey}


\begin{docTcbKey}{lines before break}{=\meta{number}}{no default, initially |2|}
  Assures that the given \meta{number} of lines of the upper box part or
  the lower box part are placed before a break happens.
\end{docTcbKey}

\clearpage
\begin{docTcbKey}[][doc updated=2017-07-05]{break at}{=\meta{length}\colOpt{/\meta{length}/\ldots/\meta{length}}}{no default, initially |0pt|}
  Defines break points at the given \meta{length} values.
  The first \meta{length} defines the (maximal) height of the first partial box,
  the second \meta{length} defines the (maximal) height of the second partial box,
  and so on. The last \meta{length} value is applied to all following partial boxes if any.
  \begin{itemize}
  \item Setting a \meta{length} to |0pt| means that the naturally available
        space is used for breaking.
  \item Setting a \meta{length} to a negative value means that
        the sum of this negative value and the naturally available space is used
        for breaking (boxes will shrink in height).
        Note that before version 4.10 negative values were treated like |0pt|.
  \end{itemize}
\begin{dispExample}
% \usepackage{multicol,lipsum}
\begin{multicols}{3}\footnotesize
Breakable boxes inside a |multicols| environment need special attendance.
They are broken by default at |\textheight|.
The |break at| option can be used to insert better break points by hand.
\begin{tcolorbox}[enhanced jigsaw,size=small,vfill before first,
  colframe=red,colback=yellow!10!white,before title=\raggedright,
  title={Broken box inside a |multicols| environment},fonttitle=\bfseries,
  enforce breakable,% use only breakable in the real world!
  pad at break=1mm,break at=3cm/6.3cm ]
\lipsum[1]
\end{tcolorbox}
\refKey{/tcb/height fixed for} may also be considered for |multicols| environments.
\end{multicols}
\end{dispExample}
\end{docTcbKey}

\enlargethispage*{1cm}

\begin{docTcbKey}{enlargepage}{=\meta{length}\colOpt{/\meta{length}/\ldots/\meta{length}}}{no default, initially |0pt|}
  Inserts a |\enlargethispage|\marg{length} to the pages of the break sequence,
  i.\,e.\ allows one to enlarge (or shrink) partial boxes. The first \meta{length} is applied
  to the first partial box, the second \meta{length} is applied
  to the second partial box, and so on. The last \meta{length} value is applied
  to all following partial boxes if any. Note that floating boxes will not be enlarged.
\begin{dispListing}
\begin{tcolorbox}[breakable,enlargepage=0mm/\baselineskip/2\baselineskip/0mm,...
\end{dispListing}
  The example code enlarged the second partial box by one line, the third
  partial box by two lines, and all following parts are not enlarged.
  \begin{marker}
  If an automated page break occures before the first partial box, the
  page enlargement is applied to the page before the first partial box \emph{and}
  again to the page of the first partial box. Insert a manual break to prevent this.\\
  In general, |enlargepage| should be used at the final stage of a document
  for fine-tuning only.
  \end{marker}
\end{docTcbKey}

\clearpage
\begin{docTcbKey}{enlargepage flexible}{=\meta{length}}{no default, initially |0pt|}
  This allows an automated page enlargement for up to \meta{length}.
  The algorithm can use this to avoid breaking a box, if there is anough room
  after enlargement. Also, the \emph{last} partial box of a break sequence
  may be enlarged to avoid further breaking.\\
  Note that this potential enlargement is \emph{additive} to settings of
  \refKey{/tcb/enlargepage}.
  But \refKey{/tcb/enlargepage flexible} overwrites settings of
  \refKey{/tcb/pad before break*} or \refKey{/tcb/pad at break*}.
\begin{dispListing}
% The following setting hinders orphan lines for the last partial box
\tcbset{enlargepage flexible=\baselineskip}
\end{dispListing}
\end{docTcbKey}


\begin{docTcbKey}[][doc new=2014-12-15]{compress page}{\colOpt{=\meta{option}}}{default |all|, initially |baselineskip|}
  This option controls the space management on the page which contains the
  unbroken box or the first part of a \emph{break sequence}.
  Feasible \meta{option} values are:
  \begin{itemize}
  \item\docValue{all} (default value):
    All shrinkable glue on the page is potentially used for the
    unbroken box or the first part of a \emph{break sequence}. Thus, all
    vertical spaces on the page will potentially be reduced to their
    minimal values.
  \item\docValue{baselineskip} (initial value):
    Shrinkable glue up to one |\baselineskip| on the page is potentially used for the
    unbroken box or the first part of a \emph{break sequence}.
  \item\docValue{none}:
    The break algorithm respects the target size of the given glue values
    on the page. This was the inital value before version |3.34|.
  \end{itemize}
  \begin{marker}
  Note that the box \emph{content} is not influenced by this option.
  \end{marker}
\end{docTcbKey}


\begin{docTcbKey}{shrink break goal}{=\meta{length}}{no default, initially |0pt|}
  This is an emergency parameter if the break algorithm produces unpleasant
  breaks.
  It shrinks the goal height of the current box part by \meta{length}
  which may result in smaller boxes. Never use negative values.
  \emph{Usually, this option will never be needed at all.}
\end{docTcbKey}



\begin{docTcbKey}[][doc new=2020-10-09]{use color stack}{\colOpt{=true\textbar false}}{default |true|, initially |false|}
  Depending on the \LaTeX\ engine and loaded packages, if your text contains some
  color changing commands, your color may not survive the break to the next box.
  For some engines, there is support for additional color stacks which
  allow colors to survive breaks. Such an color stack  can be activated
  by \refKey{/tcb/use color stack} with help of the |pdfcol| package.
  This can be done globally or per box.
  \begin{marker}
  Note that activating \refKey{/tcb/use color stack} inserts a color command with a \emph{whatsit}
  at the begin of the upper part and of the lower part of a \refEnv{tcolorbox}.
  This \emph{may} add additional vertical space, e.g. if your box text starts
  with a list like \emph{enumerate}!
  \end{marker}
  \begin{itemize}
  \item pdf\TeX: color stacks supported.
  \item Lua\TeX: color stacks supported, but you should consider loading the
     |luacolor| package \emph{instead} which avoids the spacing problem.
  \item Xe\TeX: color stacks not supported (yet?). From hearsay,
    with the |fontspec| package, you may use |\addfontfeatures{Color=mycolor}|
    to add a font color which survives the break.
  \end{itemize}
  If |pdfcol| cannot initialize an additional color stack for the used engine,
  \refKey{/tcb/use color stack} is silently ignored.

\clearpage
\begin{dispExample}
% \usepackage{multicol,lipsum}
\begin{multicols}{2}\footnotesize
Breakable box without color stack.
\begin{tcolorbox}[enhanced jigsaw,
  size=small, colframe=gray, colback=yellow!10!white, colupper=blue,
  enforce breakable,% use only breakable in the real world!
  vfill before first, pad at break=1mm, break at=3.3cm ]
    \begin{itemize}\item Some blue text.\end{itemize}
    {\color{red}\itshape\lipsum[2]}\par
    More blue text.
\end{tcolorbox}
Text after box.
\end{multicols}
\end{dispExample}

We do again with \refKey{/tcb/use color stack}. Observe the additional spacing
at the begin of the box:

\begin{dispExample}
% \usepackage{multicol,lipsum}
\begin{multicols}{2}\footnotesize
Breakable box with color stack.
\begin{tcolorbox}[enhanced jigsaw, use color stack,
  size=small, colframe=gray, colback=yellow!10!white, colupper=blue,
  enforce breakable,% use only breakable in the real world!
  vfill before first, pad at break=1mm, break at=3.3cm ]
    \begin{itemize}\item Some blue text.\end{itemize}
    {\color{red}\itshape\lipsum[2]}\par
    More blue text.
\end{tcolorbox}
Text after box.
\end{multicols}
\end{dispExample}

\end{docTcbKey}




\clearpage
\subsection{Option Keys for the Break Appearance}

\begin{docTcbKey}{toprule at break}{=\meta{length}}{no default, initially \texttt{0.5mm}}
  Sets the line width of the top rule to \meta{length} \emph{if} the box is \refKey{/tcb/breakable}.
  In this case, it is applied to \emph{middle} and \emph{last} parts in a
  break sequence. Note that \refKey{/tcb/toprule} overwrites this value
  if used afterwards.
\end{docTcbKey}


\begin{docTcbKey}{bottomrule at break}{=\meta{length}}{no default, initially \texttt{0.5mm}}
  Sets the line width of the bottom rule to \meta{length} \emph{if} the box is \refKey{/tcb/breakable}.
  In this case, it is applied to \emph{first} and \emph{middle} parts in a
  break sequence. Note that \refKey{/tcb/bottomrule} overwrites this value
  if used afterwards.
\end{docTcbKey}


\begin{docTcbKey}{topsep at break}{=\meta{length}}{no default, initially \texttt{0mm}}
  Additional vertical space of \meta{length} which is added at the top of
  \emph{middle} and \emph{last} parts in a break sequence. In general,
  it is not advisable to change this value if these parts start with a rule or a title.
\end{docTcbKey}

\begin{docTcbKey}{bottomsep at break}{=\meta{length}}{no default, initially \texttt{0mm}}
  Additional vertical space of \meta{length} which is added at the bottom of
  \emph{first} and \emph{middle} parts in a break sequence.
  In general, it is not advisable to change this value if these parts end with a rule.
\end{docTcbKey}

\begin{docTcbKey}{pad before break}{=\meta{length}}{style, no default, initially \texttt{3.5mm}}
  Sets the total amount of vertical space after the text content and before the
  break point to \meta{length}. This style sets \refKey{/tcb/toprule at break} to |0pt|
  and changes \refKey{/tcb/topsep at break} as required.
  In general, it is not advisable to change this value if the
  \emph{middle} and \emph{last} parts in a break sequence start with a rule or a title.
\end{docTcbKey}

\begin{docTcbKey}{pad before break*}{=\meta{length}}{style, no default}
  Sets \refKey{/tcb/pad before break} to \meta{length} and
  \refKey{/tcb/enlargepage flexible} to an appropriate value such that
  empty closing frames are avoided.
\end{docTcbKey}

\begin{docTcbKey}{pad after break}{=\meta{length}}{style, no default, initially \texttt{3.5mm}}
  Sets the total amount of vertical space after the break point and before the
  text content to \meta{length}. This style sets \refKey{/tcb/bottomrule at break} to |0pt|
  and changes \refKey{/tcb/bottomsep at break} as required.
  In general, it is not advisable to change this value if the
  \emph{first} and \emph{middle} parts in a break sequence end with a rule.
\end{docTcbKey}

\begin{docTcbKey}{pad at break}{=\meta{length}}{style, no default, initially \texttt{3.5mm}}
  Abbreviation for setting \meta{length} to \refKey{/tcb/pad before break}
  and \refKey{/tcb/pad after break}.
\end{docTcbKey}

\enlargethispage*{5mm}

\begin{docTcbKey}{pad at break*}{=\meta{length}}{style, no default}
  Sets \refKey{/tcb/pad at break} to \meta{length} and
  \refKey{/tcb/enlargepage flexible} to an appropriate value such that
  empty closing frames are avoided.
\end{docTcbKey}

\begin{dispListing}
% \usepackage{lipsum}  % preamble
\tcbset{colback=red!5!white,colframe=red!75!black,fonttitle=\bfseries}

\begin{tcolorbox}[enhanced jigsaw,breakable,pad at break*=0mm,
  title={For this box, the pad space at the break point is set to 0mm}]
  \lipsum[1-2]
\end{tcolorbox}
\end{dispListing}
{\tcbusetemp}


\begin{marker}
\refKey{/tcb/pad at break} or \refKey{/tcb/pad at break*}
should be used as very last option in an option list, because
they adapt other settings.
\end{marker}


\begin{marker}
Also see \refKey{/tcb/enlarge top at break by}
and \refKey{/tcb/enlarge bottom at break by}.
\end{marker}


\begin{docTcbKey}{height fixed for}{=\meta{part}}{no default, initially |none|}
  When certain amount of space is available for a partial box of a
  break sequence, the partial box typically is smaller than this space
  (depending on the box content). For given \meta{part}(s), the height can be
  set to all available space.
  \begin{itemize}
  \item\docValue{none}: Every partial |tcolorbox| is set with its natural height.
  \item\docValue{first}: The \emph{first} partial box is set to a height which matches the available space.
  \item\docValue{middle}: All \emph{middle} partial boxes are set to a height which matches the available space.
  \item\docValue{last}: The \emph{last} partial box is set to a height which matches
    the available space.
  \item\docValue{first and middle}: The \emph{first} and
    all \emph{middle} partial boxes are set to a height which matches the available space.
  \item\docValue{middle and last}: All \emph{middle} partial boxes and the \emph{last} partial box
    are set to a height which matches the available space.
  \item\docValue{all}: All partial boxes are set to a height which matches the available space.
  \end{itemize}
\begin{marker}
  If the box keeps unbroken, this option is not applied.
  See \refKey{/tcb/height} for setting a fixed height for unbroken boxes.
  See \refKey{/tcb/height fill} for giving unbroken boxes maximum height.
\end{marker}
\end{docTcbKey}


\begin{docTcbKey}{vfill before first}{\colOpt{=true\textbar false}}{default |true|, initially |false|}
  Inserts a |\vfill| at the begin of the \emph{first} partial box to move this
  partial box to the end of the current page. This may be used as an alternative
  to \refKey{/tcb/height fixed for}|=|\docValue{first} to get justified
  columns or pages. The |\vfill| is not inserted, if the box gets not
  actually broken.
\end{docTcbKey}


\begin{docTcbKey}[][doc new=2017-03-20]{segmentation at break}{\colOpt{=true\textbar false}}{default |true|, initially |true|}
  If a breakable box contains an \emph{upper part} and a \emph{lower part} and
  the break happens at the \emph{segmentation} between both parts, then
  \begin{itemize}
  \item the segmenation line (or similar) is drawn as first element of the
    partial box containing the \emph{lower part}, if \refKey{/tcb/segmentation at break}
    is set to be |true|.
  \item the segmenation line (or similar) is not drawn at all, if
    \refKey{/tcb/segmentation at break} is set to be |false|.
    This may be preferable for skins like \refSkin{bicolor}, \refSkin{tile},
    or \refSkin{beamer}.
  \end{itemize}
\end{docTcbKey}


\clearpage
\subsection{Extra Options for Partial Boxes}\label{subsec:extras}


\begin{docTcbKey}[][doc new=2015-07-16]{extras}{=\marg{options}}{no default, initially unset}
  Adds |tcolorbox| \meta{options} to every box of a \emph{break sequence}
  after skin settings are done. This is quite late in box processing.
  Geometry and break settings should \emph{not be used} here, because they
  will either be ignored or have unexpected negative results. But it is possible
  to change most colors, skin effects, shadows, borders, frame code, etc.
  Note that using \refKey{/tcb/extras} for every box is very seldom an
  advantage over setting the options directly. Usually, \refKey{/tcb/extras first},
  \refKey{/tcb/extras middle}, etc.\ are sensible to apply.
\end{docTcbKey}


\begin{docTcbKey}[][doc new=2015-07-16]{no extras}{}{style, no default, initially set}
  Removes all extras if set before.
\end{docTcbKey}


\begin{docTcbKey}[][doc new=2015-07-16]{extras broken}{=\marg{options}}{no default, initially unset}
  If the box is set to be \refKey{/tcb/breakable} and \emph{is} broken actually,
  then the \meta{options} are added to every box of the \emph{break sequence}.
  \refKey{/tcb/extras} overwrites this key.
\end{docTcbKey}

\begin{docTcbKey}[][doc new=2015-07-16]{extras unbroken}{=\marg{options}}{no default, initially unset}
  If the box is set to be \refKey{/tcb/breakable} but \emph{is not} broken actually
  or if the box is set to be \refKey{/tcb/unbreakable},
  then the \meta{options} are added to the box.
  \refKey{/tcb/extras} overwrites this key.
\end{docTcbKey}

\begin{docTcbKey}[][doc new=2015-07-16]{no extras unbroken}{}{style, no default, initially set}
  Removes the unbroken extras if set before.
\end{docTcbKey}

\begin{docTcbKey}[][doc new=2015-07-16]{extras first}{=\marg{options}}{no default, initially unset}
  If the box is set to be \refKey{/tcb/breakable} and \emph{is} broken actually,
  then the \meta{options} are added to the \emph{first} box of the break sequence.
  \refKey{/tcb/extras} overwrites this key.
\end{docTcbKey}

\begin{docTcbKey}[][doc new=2015-07-16]{no extras first}{}{style, no default, initially set}
  Removes the first extras if set before.
\end{docTcbKey}

\begin{docTcbKey}[][doc new=2015-07-16]{extras middle}{=\marg{options}}{no default, initially unset}
  If the box is set to be \refKey{/tcb/breakable} and \emph{is} broken actually,
  then the \meta{options} are added to every \emph{middle} box (if any) of the break sequence.
  \refKey{/tcb/extras} overwrites this key.
\end{docTcbKey}

\begin{docTcbKey}[][doc new=2015-07-16]{no extras middle}{}{style, no default, initially set}
  Removes the middle extras if set before.
\end{docTcbKey}

\begin{docTcbKey}[][doc new=2015-07-16]{extras last}{=\marg{options}}{no default, initially unset}
  If the box is set to be \refKey{/tcb/breakable} and \emph{is} broken actually,
  then the \meta{options} are added to the \emph{last} box of the break sequence.
  \refKey{/tcb/extras} overwrites this key.
\end{docTcbKey}

\begin{docTcbKey}[][doc new=2015-07-16]{no extras last}{}{style, no default, initially set}
  Removes the last extras if set before.
\end{docTcbKey}

\begin{docTcbKey}[][doc new=2015-07-16]{extras unbroken and first}{=\marg{options}}{no default, initially unset}
  This is an abbreviation for setting
  \refKey{/tcb/extras unbroken} and
  \refKey{/tcb/extras first} together.
  \refKey{/tcb/extras} overwrites this key.
\end{docTcbKey}

\begin{docTcbKey}[][doc new=2015-07-16]{extras middle and last}{=\marg{options}}{no default, initially unset}
  This is an abbreviation for setting
  \refKey{/tcb/extras middle} and
  \refKey{/tcb/extras last} together.
  \refKey{/tcb/extras} overwrites this key.
\end{docTcbKey}

\begin{docTcbKey}[][doc new=2015-07-16]{extras unbroken and last}{=\marg{options}}{no default, initially unset}
  This is an abbreviation for setting
  \refKey{/tcb/extras unbroken} and
  \refKey{/tcb/extras last} together.
  \refKey{/tcb/extras} overwrites this key.
\end{docTcbKey}

\clearpage

\begin{docTcbKey}[][doc new=2015-07-16]{extras first and middle}{=\marg{options}}{no default, initially unset}
  This is an abbreviation for setting
  \refKey{/tcb/extras first} and
  \refKey{/tcb/extras middle} together.
  \refKey{/tcb/extras} overwrites this key.
\end{docTcbKey}


\begin{docTcbKey}[][doc new=2018-07-26]{extras title after break}{=\marg{options}}{no default, initially unset}
  If the box has a \refKey{/tcb/title after break}, then the \meta{options}
  are added for all titles after the first break, i.e.\ all middle and last.
  The color, font, and alignment of titles after break can be adapted choosing
  \meta{options}, e.g.\ by \refKey{/tcb/coltitle}, \refKey{/tcb/fonttitle},
  \refKey{/tcb/halign title}.
  Note that \refKey{/tcb/colbacktitle} has to be placed into
  \refKey{/tcb/extras middle and last}.
\end{docTcbKey}

\begin{docTcbKey}[][doc new=2018-07-26]{no extras title after break}{}{style, no default, initially set}
  Removes the title after break extras if set before.
\end{docTcbKey}

\bigskip

\begin{exdispExample}{extras}
% \usepackage{lipsum,multicol}
% \usetikzlibrary{decorations.pathmorphing}
% \tcbuselibrary{skins}
\newtcolorbox{mybox}[1][]{
  tile,
  colback=green!7,coltitle=blue!50!black,colbacktitle=blue!5,
  center title,
  toprule=1.25mm,bottomrule=1.25mm,
  extras unbroken and first={
    borderline north={0.25mm}{0.5mm}{blue,decoration={zigzag,amplitude=0.5mm},decorate}},
  extras unbroken and last={
    borderline south={0.25mm}{0.5mm}{blue,decoration={zigzag,amplitude=0.5mm},decorate}},
  #1
}

\begin{mybox}[title=My unbroken box]
\lipsum[1]
\end{mybox}

\begin{multicols}{3}
  \begin{mybox}[title=My broken box,
    enforce breakable,% use only breakable in the real world!
    break at=4.2cm,pad at break=2mm,
    height fixed for=first and middle,  ]
  \lipsum[2]
  \end{mybox}
\end{multicols}
\end{exdispExample}




\clearpage
\subsection{Breakable boxes and the \texttt{multicol} package}\label{subsec:multicol}
\begin{marker}
With version 4.10, the algorithm for detecting the available height
for a |tcolorbox| inside a |multicol| environment was improved with help
of Frank Mittelbach. This change \emph{may} impact existing user
code which \emph{may} have to be adapted.
\end{marker}

\begin{multicols}{2}
\begin{tcolorbox}[enhanced jigsaw,size=small,breakable,colback=yellow!10!white,
  colframe=red!50!white,break at=3cm,height fixed for=all]
Unbreakable |tcolorbox|es can be used without special care inside a
|multicols| environment from the |multicol| package \cite{mittelbach:multicol}.

Since version 3.10, a breakable |tcolorbox| detects, if it is used inside
a |multicols| environment. But choosing break points for a breakable box
cannot be done by the balancing routine of |multicols|. By default, boxes
will break at maximum column height. To get pleasant results, use the
\refKey{/tcb/break at} and \refKey{/tcb/height fixed for} options.
\end{tcolorbox}
\end{multicols}

\enlargethispage{\baselineskip}
\begin{dispListing}
% \usepackage{lipsum,multicol}  % preamble
\footnotesize
\begin{multicols}{2}
  \lipsum[1]
  \begin{tcolorbox}[enhanced jigsaw,breakable,size=title,
    colback=red!5!white,colframe=red!75!black,fonttitle=\bfseries,
    title=My breakable box,pad at break=1mm, break at=-\baselineskip/0pt ]
  \lipsum[2-4]
  \end{tcolorbox}
  \lipsum[4]
\end{multicols}
\end{dispListing}
{\tcbusetemp}

\clearpage

\begin{multicols}{2}
\small
This example is already set inside a |multicols| environment.
This time, a \emph{middle} part has full column height (here |\textheight|).
\refKey{/tcb/height fixed for} is used to spread this box part over the full
height to align with neighboring columns.
\begin{dispListing}
% \usepackage{lipsum,multicol}
\lipsum[1]
\begin{tcolorbox}[enhanced jigsaw,
  breakable,
  size=title,
  colback=red!5!white,
  colframe=red!75!black,
  fonttitle=\bfseries,
  title=My breakable box,
  pad at break=2mm,
  break at=-\baselineskip/0pt,
  height fixed for=middle ]
\lipsum[2-7]
\end{tcolorbox}
\lipsum[8]
\end{dispListing}
{\tcbusetemp}
\end{multicols}


The following example has a |\tcolorbox| which fills the |\multicols|
environment completely. Here, \refKey{/tcb/height fixed for} is used
to give all three columns the full height.
Note that the appropriate \refKey{/tcb/break at} value is not computed
automatically but set manually.

\begin{dispListing}
% \usepackage{lipsum,multicol}  % preamble
\small
\begin{multicols}{3}
  \begin{tcolorbox}[enhanced jigsaw,breakable,size=small,
    colback=red!5!white,colframe=red!75!black,fonttitle=\bfseries,
    title=My breakable box,pad at break=2mm,drop fuzzy shadow,
    height fixed for=all, break at=11.4cm ]
  \lipsum[1-3]
  \end{tcolorbox}
\end{multicols}
\end{dispListing}
{\tcbusetemp}

\clearpage
\subsection{Break Point Insertion}\label{subsec:breakpoints}

\begin{docCommand}[doc new=2017-07-05]{tcbbreak}{}
  A \emph{breakable} box is not broken, if there is enough
  space on the current page or column.
  Therefore, typical penalty insertion with
  |\break|, |\pagebreak|, |\columnbreak|, \ldots \emph{may} only work as
  expected, if the box is broken at least into two parts
  \emph{without} inserting the penalties.\par\smallskip
  To \emph{force} a page or column break, \refCom{tcbbreak}
  starts a new paragraph and inserts an insane tall rule which causes a
  break and which is immediately discarded. You may ignore this technical
  information and just use it as you would use |\pagebreak|.\par\smallskip
  For an \emph{unbreakable box}, \refCom{tcbbreak} is identical to insert |\par|,
  i.e.\ it just starts a new paragraph.\par\smallskip
  Also see \refKey{/tcb/break at} for defining height dependend breaks.

\begin{dispListing}
% \usepackage{lipsum,multicol}  % preamble
\begin{multicols}{3}
  \begin{tcolorbox}[breakable,enhanced jigsaw,size=small,
    colback=red!5!white,colframe=red!75!black,fonttitle=\bfseries,
    title=Break into parts
  ]
  First part\tcbbreak
  Second part\tcbbreak
  Third part
  \end{tcolorbox}
\end{multicols}

\begin{multicols}{3}
  \begin{tcolorbox}[enhanced jigsaw,size=small,
    colback=red!5!white,colframe=red!75!black,fonttitle=\bfseries,
    title=You shall not break
  ]
  First part\tcbbreak
  Second part\tcbbreak
  Third part
  \end{tcolorbox}
\end{multicols}

\end{dispListing}
{\tcbusetemp}

\end{docCommand}



\clearpage
\subsection{Break Sequence for the Skins}\label{subsec:breaksequence}
The following diagrams document the \emph{break sequence} for different
skins. Depending on the main skin of a |tcolorbox|, the actual skins of
the \emph{break sequence} parts are displayed.

\def\tcbbreakskininto#1#2#3#4#5{%
\begin{center}\begin{tikzpicture}
\tcbset{width=7cm,colframe=Navy,colback=AliceBlue,fonttitle=\bfseries,
  watermark color=AliceBlue!85!Navy,#5
  }
\node[above] (unbroken) at (0,0) {\begin{tcolorbox}[title=Unbroken Box,skin=#1,watermark text=unbroken,height=3.8cm]
\texttt{skin=#1}
\end{tcolorbox}};
\node[above] (first) at (8.7,2.4) {\begin{tcolorbox}[title=Broken Boxes,skin=#2,watermark text=first,height=1.4cm]
\texttt{skin=#2}
\end{tcolorbox}};
\node[above] (middle) at (8.7,1.2) {\begin{tcolorbox}[skin=#3,watermark text=middle,height=1cm]
\texttt{skin=#3}
\end{tcolorbox}};
\node[above] (last) at (8.7,0) {\begin{tcolorbox}[skin=#4,watermark text=last,height=1cm]
\texttt{skin=#4}
\end{tcolorbox}};
\path[draw=FireBrick,line width=2pt,->] (unbroken) edge (first.west) edge (middle.west) edge (last.west);
\end{tikzpicture}\end{center}}

\tcbbreakskininto{standard}{standard}{standard}{standard}{watermark text/.style={}}
\tcbbreakskininto{standard jigsaw}{standard jigsaw}{standard jigsaw}{standard jigsaw}{watermark text/.style={}}
\tcbbreakskininto{spartan}{spartan}{spartan}{spartan}{}
\clearpage
\tcbbreakskininto{enhanced}{enhancedfirst}{enhancedmiddle}{enhancedlast}{}
\tcbbreakskininto{enhancedfirst}{enhancedfirst}{enhancedmiddle}{enhancedmiddle}{}
\tcbbreakskininto{enhancedmiddle}{enhancedmiddle}{enhancedmiddle}{enhancedmiddle}{}
\tcbbreakskininto{enhancedlast}{enhancedmiddle}{enhancedmiddle}{enhancedlast}{}
\clearpage
\tcbbreakskininto{enhanced jigsaw}{enhancedfirst jigsaw}{enhancedmiddle jigsaw}{enhancedlast jigsaw}{}
\tcbbreakskininto{enhancedfirst jigsaw}{enhancedfirst jigsaw}{enhancedmiddle jigsaw}{enhancedmiddle jigsaw}{}
\tcbbreakskininto{enhancedmiddle jigsaw}{enhancedmiddle jigsaw}{enhancedmiddle jigsaw}{enhancedmiddle jigsaw}{}
\tcbbreakskininto{enhancedlast jigsaw}{enhancedmiddle jigsaw}{enhancedmiddle jigsaw}{enhancedlast jigsaw}{}
\clearpage
{\tcbset{borderline={2pt}{0pt}{black!10!white}}%
\tcbbreakskininto{empty}{emptyfirst}{emptymiddle}{emptylast}{}
\tcbbreakskininto{emptyfirst}{emptyfirst}{emptymiddle}{emptymiddle}{}
\tcbbreakskininto{emptymiddle}{emptymiddle}{emptymiddle}{emptymiddle}{}
\tcbbreakskininto{emptylast}{emptymiddle}{emptymiddle}{emptylast}{}
}
\clearpage
\tcbbreakskininto{bicolor}{bicolorfirst}{bicolormiddle}{bicolorlast}{bicolor}
\tcbbreakskininto{bicolorfirst}{bicolorfirst}{bicolormiddle}{bicolormiddle}{bicolor}
\tcbbreakskininto{bicolormiddle}{bicolormiddle}{bicolormiddle}{bicolormiddle}{bicolor}
\tcbbreakskininto{bicolorlast}{bicolormiddle}{bicolormiddle}{bicolorlast}{bicolor}
\clearpage
\tcbbreakskininto{bicolor jigsaw}{bicolorfirst jigsaw}{bicolormiddle jigsaw}{bicolorlast jigsaw}{bicolor jigsaw}
\tcbbreakskininto{bicolorfirst jigsaw}{bicolorfirst jigsaw}{bicolormiddle jigsaw}{bicolormiddle jigsaw}{bicolor jigsaw}
\tcbbreakskininto{bicolormiddle jigsaw}{bicolormiddle jigsaw}{bicolormiddle jigsaw}{bicolormiddle jigsaw}{bicolor jigsaw}
\tcbbreakskininto{bicolorlast jigsaw}{bicolormiddle jigsaw}{bicolormiddle jigsaw}{bicolorlast jigsaw}{bicolor jigsaw}
\clearpage
\tcbbreakskininto{tile}{tilefirst}{tilemiddle}{tilelast}{tile,colbacktitle=Navy}
\tcbbreakskininto{tilefirst}{tilefirst}{tilemiddle}{tilemiddle}{tile,colbacktitle=Navy}
\tcbbreakskininto{tilemiddle}{tilemiddle}{tilemiddle}{tilemiddle}{tile,colbacktitle=Navy}
\tcbbreakskininto{tilelast}{tilemiddle}{tilemiddle}{tilelast}{tile,colbacktitle=Navy}
\clearpage
\tcbbreakskininto{beamer}{beamerfirst}{beamermiddle}{beamerlast}{beamer}
\tcbbreakskininto{beamerfirst}{beamerfirst}{beamermiddle}{beamermiddle}{beamer}
\tcbbreakskininto{beamermiddle}{beamermiddle}{beamermiddle}{beamermiddle}{beamer}
\tcbbreakskininto{beamerlast}{beamermiddle}{beamermiddle}{beamerlast}{beamer}
\clearpage
\tcbbreakskininto{widget}{widgetfirst}{widgetmiddle}{widgetlast}{widget}
\tcbbreakskininto{widgetfirst}{widgetfirst}{widgetmiddle}{widgetmiddle}{widget}
\tcbbreakskininto{widgetmiddle}{widgetmiddle}{widgetmiddle}{widgetmiddle}{widget}
\tcbbreakskininto{widgetlast}{widgetmiddle}{widgetmiddle}{widgetlast}{widget}
\tcbbreakskininto{draft}{draft}{draft}{draft}{draft}
\clearpage
\tcbbreakskininto{freelance}{freelancefirst}{freelancemiddle}{freelancelast}{}
\tcbbreakskininto{freelancefirst}{freelancefirst}{freelancemiddle}{freelancemiddle}{}
\tcbbreakskininto{freelancemiddle}{freelancemiddle}{freelancemiddle}{freelancemiddle}{}
\tcbbreakskininto{freelancelast}{freelancemiddle}{freelancemiddle}{freelancelast}{}




\clearpage
\subsection{Break by Hand (Faked Break)}

\begin{marker}
See \Vref{subsec:multicol} for \emph{real} column breaks.
\end{marker}

Since the appearance of broken boxes is done by skins, it is quite easy
to 'fake a break'. For this, you actually don't need the
\mylib{breakable} library at
all.

\begin{dispExample}
\tcbset{enhanced,equal height group=fakedbreak,
  colback=LightGreen,colframe=DarkGreen,
  width=(\linewidth-6mm)/3,nobeforeafter,
  left=1mm,right=1mm,top=1mm,bottom=1mm,middle=1mm}
%
\begin{tcolorbox}[title=My broken box,skin=enhancedfirst]
This is a box which breaks from one column to another
\end{tcolorbox}\hfill
\begin{tcolorbox}[skin=enhancedmiddle]
column. I am sorry to say that this is a trick.
Nevertheless, you may use this trick for your
\end{tcolorbox}\hfill
\begin{tcolorbox}[skin=enhancedlast]
own purposes.
\end{tcolorbox}
\end{dispExample}


